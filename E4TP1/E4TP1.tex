%% LyX 2.2.3 created this file.  For more info, see http://www.lyx.org/.
%% Do not edit unless you really know what you are doing.
\documentclass[english]{article}
\usepackage[T1]{fontenc}
\usepackage[latin9]{inputenc}
\usepackage{geometry}
\geometry{verbose,tmargin=3cm,bmargin=3cm,lmargin=3cm,rmargin=3cm,headheight=3cm,headsep=3cm}
\usepackage{float}

\makeatletter

%%%%%%%%%%%%%%%%%%%%%%%%%%%%%% LyX specific LaTeX commands.
%% Because html converters don't know tabularnewline
\providecommand{\tabularnewline}{\\}

\makeatother

\usepackage{babel}
\begin{document}

\title{Electr�nica III TP1}
\maketitle

\section{Ejercicio 4}

\begin{figure}[H]
\begin{centering}
\begin{tabular}{|c|c|c|c|c|c|c|c|}
\hline 
$x_{1}$ & $x_{2}$ & $x_{3}$ & $x_{4}$ & $f_{1}$ & $f_{2}$ & $f_{3}$ & $f_{4}$\tabularnewline
\hline 
\hline 
0 & 0 & 0 & 0 & 0 & 0 & 0 & 1\tabularnewline
\hline 
0 & 0 & 0 & 1 & 1 & 1 & 1 & 1\tabularnewline
\hline 
0 & 0 & 1 & 0 & 1 & 1 & 1 & 0\tabularnewline
\hline 
0 & 0 & 1 & 1 & 1 & 1 & 0 & 1\tabularnewline
\hline 
0 & 1 & 0 & 0 & 1 & 1 & 0 & 0\tabularnewline
\hline 
0 & 1 & 0 & 1 & 1 & 0 & 1 & 1\tabularnewline
\hline 
0 & 1 & 1 & 0 & 1 & 0 & 1 & 0\tabularnewline
\hline 
0 & 1 & 1 & 1 & 1 & 0 & 0 & 1\tabularnewline
\hline 
1 & 0 & 0 & 0 & 1 & 0 & 0 & 0\tabularnewline
\hline 
1 & 0 & 0 & 1 & 0 & 1 & 1 & 1\tabularnewline
\hline 
1 & 0 & 1 & 0 & 0 & 1 & 1 & 0\tabularnewline
\hline 
1 & 0 & 1 & 1 & 0 & 1 & 0 & 1\tabularnewline
\hline 
1 & 1 & 0 & 0 & 0 & 1 & 0 & 0\tabularnewline
\hline 
1 & 1 & 0 & 1 & 0 & 0 & 1 & 1\tabularnewline
\hline 
1 & 1 & 1 & 0 & 0 & 0 & 1 & 0\tabularnewline
\hline 
1 & 1 & 1 & 1 & 0 & 0 & 0 & 1\tabularnewline
\hline 
\end{tabular}
\par\end{centering}
\caption{Complemento a 2 de los bits de entrada}
\end{figure}

Si escribimos cada bit de salida en funci�n de los mint�rminos de
los bits de entrada, nos quedan las siguientes ecuaci�nes:
\begin{center}
$f_{1}(m_{i})=m_{1}+m_{2}+m_{3}+m_{4}+m_{5}+m_{6}+m_{7}+m_{8}$
\par\end{center}

\begin{center}
$f_{2}(m_{i})=m_{1}+m_{2}+m_{3}+m_{4}+m_{9}+m_{10}+m_{11}+m_{12}$
\par\end{center}

\begin{center}
$f_{3}(m_{i})=m_{1}+m_{2}+m_{5}+m_{6}+m_{9}+m_{10}+m_{13}+m_{14}$
\par\end{center}

\begin{center}
$f_{4}(m_{i})=m_{0}+m_{1}+m_{3}+m_{5}+m_{7}+m_{9}+m_{11}+m_{13}+m_{15}$
\par\end{center}

Remplazando los valores de cada mintermino, quedan las siguientes
formulas:
\begin{center}
$f_{1}(x_{1};x_{2};x_{3};x_{4})=\bar{x_{1}}\bar{x_{2}}\bar{x_{3}}x_{4}+\bar{x_{1}}\bar{x_{2}}x_{3}\bar{x_{4}}+\bar{x_{1}}\bar{x_{2}}x_{3}x_{4}+\bar{x_{1}}x_{2}\bar{x_{3}}\bar{x_{4}}+\bar{x_{1}}x_{2}\bar{x_{3}}x_{4}+\bar{x_{1}}x_{2}x_{3}\bar{x_{4}}+\bar{x_{1}}x_{2}x_{3}x_{4}+x_{1}\bar{x_{2}}\bar{x_{3}}\bar{x_{4}}$
\par\end{center}

\begin{center}
$f_{2}(x_{1};x_{2};x_{3};x_{4})=\bar{x_{1}}\bar{x_{2}}\bar{x_{3}}x_{4}+\bar{x_{1}}\bar{x_{2}}x_{3}\bar{x_{4}}+\bar{x_{1}}\bar{x_{2}}x_{3}x_{4}+\bar{x_{1}}x_{2}\bar{x_{3}}\bar{x_{4}}+x_{1}\bar{x_{2}}\bar{x_{3}}x_{4}+x_{1}\bar{x_{2}}x_{3}\bar{x_{4}}+x_{1}\bar{x_{2}}x_{3}x_{4}+x_{1}x_{2}\bar{x_{3}}\bar{x_{4}}$
\par\end{center}

\begin{center}
$f_{3}(x_{1};x_{2};x_{3};x_{4})=\bar{x_{1}}\bar{x_{2}}\bar{x_{3}}x_{4}+\bar{x_{1}}\bar{x_{2}}x_{3}\bar{x_{4}}+\bar{x_{1}}x_{2}\bar{x_{3}}x_{4}+\bar{x_{1}}x_{2}x_{3}\bar{x_{4}}+x_{1}\bar{x_{2}}\bar{x_{3}}x_{4}+x_{1}\bar{x_{2}}x_{3}\bar{x_{4}}+x_{1}x_{2}\bar{x_{3}}x_{4}+x_{1}x_{2}x_{3}\bar{x_{4}}$
\par\end{center}

\begin{center}
$f_{4}(x_{1};x_{2};x_{3};x_{4})=\bar{x_{1}}\bar{x_{2}}\bar{x_{3}}\bar{x_{4}}+\bar{x_{1}}\bar{x_{2}}\bar{x_{3}}x_{4}+\bar{x_{1}}\bar{x_{2}}x_{3}x_{4}+\bar{x_{1}}x_{2}\bar{x_{3}}x_{4}+\bar{x_{1}}x_{2}x_{3}x_{4}+x_{1}\bar{x_{2}}\bar{x_{3}}x_{4}+x_{1}\bar{x_{2}}x_{3}x_{4}+x_{1}x_{2}\bar{x_{3}}x_{4}+x_{1}x_{2}x_{3}x_{4}$
\par\end{center}

Si simplificamos cada ecuaci�n, se puede llegar a las siguientes expresiones:
\begin{center}
$f_{1}(x_{1};x_{2};x_{3};x_{4})=x_{1}\bar{x_{2}}\bar{x_{3}}\bar{x_{4}}+\bar{x_{1}}(x_{2}+x_{3}+x_{4})$
\par\end{center}

\begin{center}
$f_{2}(x_{1};x_{2};x_{3};x_{4})=x_{2}\bar{x_{3}}\bar{x_{4}}+\bar{x_{2}}(x_{3}+x_{4})$
\par\end{center}

\begin{center}
$f_{3}(x_{1};x_{2};x_{3};x_{4})=x_{3}\bar{x_{4}}+\bar{x_{3}}x_{4}$
\par\end{center}

\begin{center}
$f_{4}(x_{1};x_{2};x_{3};x_{4})=x_{4}+\bar{x_{1}}\bar{x_{2}}\bar{x_{3}}$
\par\end{center}


\end{document}
