\section{\color{olive}Excercise 2: Simplification of a Maxterm Expression and its Corresponding Logical Circuit}

Having the function in maxterms $$f_1 (A,B,C,D) = \prod\left(M_0, M_1 , M_5 , M_7 , M_8 , M_{10} , M_{14} , M_{15} \right)$$ equivalent to $$f_2 (A,B,C,D) = \sum\left(m_2, m_3 , m_4 , m_6 , m_9 , m_{11} , m_{12} , m_{13} \right)$$ using minterms, can be simplify by different ways and represented using logic gates.

    \subsection{\color{purple}Simplify: Boolean Algebra}

    Using the Boolean algebra propertie $$(A+B).(A+\overline{B})=A~(1)$$ or $$(AB)+(A\overline{B})=A~(2)$$ the function could be simplify using (1): 
    \begin{eqnarray*}
        f_1 (A,B,C,D) &= &(A+B+C+D).(A+B+C+\overline{D}).(A+\overline{B}+C+\overline{D}).(A+\overline{B}+\overline{C}+\overline{D}).\\
        &&(\overline{A}+B+C+D).(\overline{A}+B+\overline{C}+D).(\overline{A}+\overline{B}+\overline{C}+D).(\overline{A}+\overline{B}+\overline{C}+\overline{D}) \\
        &=&(A+B+C).(A+\overline{B}+\overline{D}).(\overline{A}+B+D).(\overline{A}+\overline{B}+\overline{C})
    \end{eqnarray*}
    Which in minterms would be, using (2):
    \begin{eqnarray*}
        f_2(A,B,C,D) &= &(\overline{A}\overline{B}C\overline{D})+(\overline{A}\overline{B}CD)+(\overline{A}B\overline{C}\overline{D})+(\overline{A}BC\overline{D})+\\
        &&(A\overline{B}\overline{C}D)+(A\overline{B}CD)+(AB\overline{C}\overline{D})+(AB\overline{C}D)\\
        &=&(A\overline{B}D)+(\overline{A}B\overline{D})+(\overline{A}\overline{B}C)+(AB\overline{C})
    \end{eqnarray*}

    \subsection{\color{purple}Simplify: Karnaugh Map}

    Karnaugh map is a easier way to simplify logic experesion when the functions are too complex or too large to handle, cause Karnaugh map gives a more representative view for a faster analisis for it to simplify. 

    If the simplification is done with minterms, the groups should be of $1$, adding each group in case there is more than 1, and in each group the independent variables would be multiplied.
    \begin{center}
        \begin{Karnaugh}
            %cada 4 es una fila, la col 3 es la 4ta columna y 3fila es la 4 fila
            \contingut{
            0,1,0,1,
            0,0,1,1,
            1,1,0,0,
            1,0,1,0} 
            \implicant{6}{14}{red}
            \implicant{3}{7}{green}
            \implicant{12}{8}{orange}
            \implicantdaltbaix[3pt]{1}{9}{blue}
            %\implicantcantons[2pt]{orange}
            %\implicantcostats{4}{14}{green}
        \end{Karnaugh}
    \end{center}
    
    Now grouping the colour groups we get that the function in minterms would be:
    \begin{eqnarray*}
        f_2(A,B,C,D)&=&(A\overline{B}D)~(Red)\\
        &&+(\overline{A}B\overline{D})~(Blue)\\
        &&+(\overline{A}\overline{B}C)~(Orange)\\
        &&+(AB\overline{C})~(Green)
    \end{eqnarray*}
 
    The same method could be done with maxterms; grouping $O$, multipling groups in case there is more than 1, and in each group the independent variables would be added.
    
    \subsection{\color{purple}Logic Circuit: AND, OR and NOT}

    Using the logic gates AND, OR and NOT the simplify version of the function could be represented in the figure below:

    \begin{figure}[htb!]
        \centering
        \includegraphics[scale=0.45]{E2TP1/normallogic.png}
        \caption{\color{cyan}Logic circuit using AND, OR and NOT gates}
        \label{fig:normllogic}
    \end{figure}
    
    \pagebreak

    \subsection{\color{purple}Logic Circuit: NAND}

    All the gates could be equivalent to a combination of NAND or NOR gates. Therefore, the simplify function can be drawn as the next figure:

    \begin{figure}[h!]
        \centering
        \includegraphics[scale=0.45]{E2TP1/nandlogic.png}
        \caption{\color{cyan}Logic circuit using NAND gates}
        \label{fig:nandlogic}
    \end{figure}