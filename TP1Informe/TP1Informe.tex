\documentclass[a4paper,12pt]{report}
\usepackage{graphicx}
\usepackage{color} 
\usepackage[dvipsnames]{xcolor}
\colorlet{purple}{purple}

 \usepackage{geometry} % Required to change the page size to A4
  %  \geometry{a4paper} % Set the page size to be A4 as opposed to the default US Letter

 \usepackage{mathtools, nccmath}    
 \usepackage{tikz}
 \usetikzlibrary{matrix,calc}

    %isolated term
%#1 - Optional. Space between node and grouping line. Default=0
%#2 - node
%#3 - filling color
\newcommand{\implicantsol}[3][0]{
    \draw[rounded corners=3pt, fill=#3, opacity=0.3] ($(#2.north west)+(135:#1)$) rectangle ($(#2.south east)+(-45:#1)$);
    }


%internal group
%#1 - Optional. Space between node and grouping line. Default=0
%#2 - top left node
%#3 - bottom right node
%#4 - filling color
\newcommand{\implicant}[4][0]{
    \draw[rounded corners=3pt, fill=#4, opacity=0.3] ($(#2.north west)+(135:#1)$) rectangle ($(#3.south east)+(-45:#1)$);
    }

%group lateral borders
%#1 - Optional. Space between node and grouping line. Default=0
%#2 - top left node
%#3 - bottom right node
%#4 - filling color
\newcommand{\implicantcostats}[4][0]{
    \draw[rounded corners=3pt, fill=#4, opacity=0.3] ($(rf.east |- #2.north)+(90:#1)$)-| ($(#2.east)+(0:#1)$) |- ($(rf.east |- #3.south)+(-90:#1)$);
    \draw[rounded corners=3pt, fill=#4, opacity=0.3] ($(cf.west |- #2.north)+(90:#1)$) -| ($(#3.west)+(180:#1)$) |- ($(cf.west |- #3.south)+(-90:#1)$);
}

%group top-bottom borders
%#1 - Optional. Space between node and grouping line. Default=0
%#2 - top left node
%#3 - bottom right node
%#4 - filling color
\newcommand{\implicantdaltbaix}[4][0]{
    \draw[rounded corners=3pt, fill=#4, opacity=0.3] ($(cf.south -| #2.west)+(180:#1)$) |- ($(#2.south)+(-90:#1)$) -| ($(cf.south -| #3.east)+(0:#1)$);
    \draw[rounded corners=3pt, fill=#4, opacity=0.3] ($(rf.north -| #2.west)+(180:#1)$) |- ($(#3.north)+(90:#1)$) -| ($(rf.north -| #3.east)+(0:#1)$);
}

%group corners
%#1 - Optional. Space between node and grouping line. Default=0
%#2 - filling color
\newcommand{\implicantcantons}[2][0]{
    \draw[rounded corners=3pt, opacity=.3] ($(rf.east |- 0.south)+(-90:#1)$) -| ($(0.east |- cf.south)+(0:#1)$);
    \draw[rounded corners=3pt, opacity=.3] ($(rf.east |- 8.north)+(90:#1)$) -| ($(8.east |- rf.north)+(0:#1)$);
    \draw[rounded corners=3pt, opacity=.3] ($(cf.west |- 2.south)+(-90:#1)$) -| ($(2.west |- cf.south)+(180:#1)$);
    \draw[rounded corners=3pt, opacity=.3] ($(cf.west |- 10.north)+(90:#1)$) -| ($(10.west |- rf.north)+(180:#1)$);
    \fill[rounded corners=3pt, fill=#2, opacity=.3] ($(rf.east |- 0.south)+(-90:#1)$) -|  ($(0.east |- cf.south)+(0:#1)$) [sharp corners] ($(rf.east |- 0.south)+(-90:#1)$) |-  ($(0.east |- cf.south)+(0:#1)$) ;
    \fill[rounded corners=3pt, fill=#2, opacity=.3] ($(rf.east |- 8.north)+(90:#1)$) -| ($(8.east |- rf.north)+(0:#1)$) [sharp corners] ($(rf.east |- 8.north)+(90:#1)$) |- ($(8.east |- rf.north)+(0:#1)$) ;
    \fill[rounded corners=3pt, fill=#2, opacity=.3] ($(cf.west |- 2.south)+(-90:#1)$) -| ($(2.west |- cf.south)+(180:#1)$) [sharp corners]($(cf.west |- 2.south)+(-90:#1)$) |- ($(2.west |- cf.south)+(180:#1)$) ;
    \fill[rounded corners=3pt, fill=#2, opacity=.3] ($(cf.west |- 10.north)+(90:#1)$) -| ($(10.west |- rf.north)+(180:#1)$) [sharp corners] ($(cf.west |- 10.north)+(90:#1)$) |- ($(10.west |- rf.north)+(180:#1)$) ;
}

%Empty Karnaugh map 4x4
\newenvironment{Karnaugh}%
{
\begin{tikzpicture}[baseline=(current bounding box.north),scale=0.8]
\draw (0,0) grid (4,4);
\draw (0,4) -- node [pos=0.7,above right,anchor=south west] {AB} node [pos=0.75,below left,anchor=north east] {CD} ++(135:1);
%
\matrix (mapa) [matrix of nodes,
        column sep={0.8cm,between origins},
        row sep={0.8cm,between origins},
        every node/.style={minimum size=0.3mm},
        anchor=8.center,
        ampersand replacement=\&] at (0.5,0.5)
{
                       \& |(c00)| 00         \& |(c01)| 01         \& |(c11)| 11         \& |(c10)| 10         \& |(cf)| \phantom{00} \\
|(r00)| 00             \& |(0)|  \phantom{0} \& |(1)|  \phantom{0} \& |(3)|  \phantom{0} \& |(2)|  \phantom{0} \&                     \\
|(r01)| 01             \& |(4)|  \phantom{0} \& |(5)|  \phantom{0} \& |(7)|  \phantom{0} \& |(6)|  \phantom{0} \&                     \\
|(r11)| 11             \& |(12)| \phantom{0} \& |(13)| \phantom{0} \& |(15)| \phantom{0} \& |(14)| \phantom{0} \&                     \\
|(r10)| 10             \& |(8)|  \phantom{0} \& |(9)|  \phantom{0} \& |(11)| \phantom{0} \& |(10)| \phantom{0} \&                     \\
|(rf) | \phantom{00}   \&                    \&                    \&                    \&                    \&                     \\
};
}%
{
\end{tikzpicture}
}

%Empty Karnaugh map 2x4
\newenvironment{Karnaughvuit}%
{
\begin{tikzpicture}[baseline=(current bounding box.north),scale=0.8]
\draw (0,0) grid (4,2);
\draw (0,2) -- node [pos=0.7,above right,anchor=south west] {bc} node [pos=0.7,below left,anchor=north east] {a} ++(135:1);
%
\matrix (mapa) [matrix of nodes,
        column sep={0.8cm,between origins},
        row sep={0.8cm,between origins},
        every node/.style={minimum size=0.3mm},
        anchor=4.center,
        ampersand replacement=\&] at (0.5,0.5)
{
                      \& |(c00)| 00         \& |(c01)| 01         \& |(c11)| 11         \& |(c10)| 10         \& |(cf)| \phantom{00} \\
|(r00)| 0             \& |(0)|  \phantom{0} \& |(1)|  \phantom{0} \& |(3)|  \phantom{0} \& |(2)|  \phantom{0} \&                     \\
|(r01)| 1             \& |(4)|  \phantom{0} \& |(5)|  \phantom{0} \& |(7)|  \phantom{0} \& |(6)|  \phantom{0} \&                     \\
|(rf) | \phantom{00}  \&                    \&                    \&                    \&                    \&                     \\
};
}%
{
\end{tikzpicture}
}

%Empty Karnaugh map 2x2
\newenvironment{Karnaughquatre}%
{
\begin{tikzpicture}[baseline=(current bounding box.north),scale=0.8]
\draw (0,0) grid (2,2);
\draw (0,2) -- node [pos=0.7,above right,anchor=south west] {b} node [pos=0.7,below left,anchor=north east] {a} ++(135:1);
%
\matrix (mapa) [matrix of nodes,
        column sep={0.8cm,between origins},
        row sep={0.8cm,between origins},
        every node/.style={minimum size=0.3mm},
        anchor=2.center,
        ampersand replacement=\&] at (0.5,0.5)
{
          \& |(c00)| 0          \& |(c01)| 1  \\
|(r00)| 0 \& |(0)|  \phantom{0} \& |(1)|  \phantom{0} \\
|(r01)| 1 \& |(2)|  \phantom{0} \& |(3)|  \phantom{0} \\
};
}%
{
\end{tikzpicture}
}

%Defines 8 or 16 values (0,1,X)
\newcommand{\contingut}[1]{%
\foreach \x [count=\xi from 0]  in {#1}
     \path (\xi) node {\x};
}

%Places 1 in listed positions
\newcommand{\minterms}[1]{%
    \foreach \x in {#1}
        \path (\x) node {1};
}

%Places 0 in listed positions
\newcommand{\maxterms}[1]{%
    \foreach \x in {#1}
        \path (\x) node {0};
}

%Places X in listed positions
\newcommand{\indeterminats}[1]{%
    \foreach \x in {#1}
        \path (\x) node {X};
}

    \linespread{1.2} % Line spacing
    
    \setlength\parindent{0pt} % Uncomment to remove all indentation from paragraphs
    
    \graphicspath{{/home/bzerol/VisualCode/ElectroIII/tp1-team-2/E2TP1}} % Specifies the directory where pictures are stored

%///////////////////////////////////////////////////////////////////////////////////////////////////////////////////////////////////////////////////////////////////
\begin{document}
\title{\color{magenta}\underline {Assignment $N^o 1$}}
\author{Electronica III}
\author{\color{teal}Group 2}
\date{\color{blue}\today} %ver si dejar la de today o poner fecha fija que sea August 2018
\pagenumbering{arabic}
\maketitle

%////////////////////////////////////////////////////////////////////// EXCERCISE 1 /////////////////////////////////////////////////////////////////////////////
\section{\color{olive}Excercise 1: Resolution and Range of a Fixed-Point Binary Representation}

\subsection{\color{purple}What is the Fixed-Point Binary Representation}
A fixed-point number has an integer part and a fractional part separated by a decimal point with a fixed position, as shown below:
$$ (Integer Part).(Fractional Part)$$
The integer part is formed by $n$ bits and the fractional part is formed by $m$ bits.
$$ (bit \#1\ \ \ bit \#2\ \ \ ...\ \ \ bit \#n).(bit \#1\ \ \ bit \#2\ \ \ ...\ \ \ bit \#m)$$

\subsection{\color{purple}What is Resolution and Range}
\subsubsection{\color{red}Resolution}
The resolution of a number using the fixed point representation is the smallest unit that can be handled with it.
Given a fixed-point number with $m$ fractional bits, the resolution is $2^- $$^m$. %Importante! Ver acá como puse el 2 elevado a la -m, para futuras veces.
\subsubsection{\color{red}Range}
The range is the difference between the biggest value that can be obtained with the fixed-point representation of a number with $n$ bits in the integer part and with $m$ bits in the fractional part, and the smallest number that can be represented.
%ver de poner la ecuacion matematica que permite obtener el range.

\subsection{\color{purple}Making Use of this Program}
\subsubsection{\color{orange}Input}
Three arguments must be entered through Command Line, separated by one space:
\begin{enumerate}
\item  1 (indicating that the numeric representation of the binary number is signed) or 0 (indicating that the representation is unsigned).
\item $n$: A possitive integer (indicating the number of bits that correspond to the integer part of the number, which appears before the decimal point).
\item  $m$: A possitive integer (indicating the number of bits that correspond to the fractional part of the number, which appears after the decimal point).
\end{enumerate}
{\color{cyan}For example: "0 1 1"}.

\subsubsection{\color{orange}Output}
The result of this program is the resolution and range of the number that has $n$ digits in the integer part and $m$ digits in the fractional part.

\begin{figure}[h!]
\centering
\includegraphics[scale=1]{ejemploOutput}
\caption{\color{cyan}Output corresponding to the example input $"0\ 1\ 1"$.}
\label{image output}
\end{figure}

\subsection{\color{purple}Testing the Program}
%HACEEEEEEEEEEEEEEEEEEEEEEEEEEEEEEEEEEEEEEEEEEEER ESTA PARTEEEEEEEEEEEEEEEEEEEEEEEEEE

%////////////////////////////////////////////////////////////////////// EXCERCISE 2 /////////////////////////////////////////////////////////////////////////////
\section{\color{olive}Excercise 2: Simplification of a Maxterm Expression and its Corresponding Logical Circuit}

Having the function in maxterms $$f_1 (A,B,C,D) = \prod\left(M_0, M_1 , M_5 , M_7 , M_8 , M_{10} , M_{14} , M_{15} \right)$$ equivalent to $$f_2 (A,B,C,D) = \sum\left(m_2, m_3 , m_4 , m_6 , m_9 , m_{11} , m_{12} , m_{13} \right)$$ using minterms, can be simplify by different ways and represented using logic gates.

    \subsection{\color{purple}Simplify: Boolean Algebra}

    Using the Boolean algebra propertie $$(A+B).(A+\overline{B})=A~(1)$$ or $$(AB)+(A\overline{B})=A~(2)$$ the function could be simplify using (1): 
    \begin{eqnarray*}
        f_1 (A,B,C,D) &= &(A+B+C+D).(A+B+C+\overline{D}).(A+\overline{B}+C+\overline{D}).(A+\overline{B}+\overline{C}+\overline{D}).\\
        &&(\overline{A}+B+C+D).(\overline{A}+B+\overline{C}+D).(\overline{A}+\overline{B}+\overline{C}+D).(\overline{A}+\overline{B}+\overline{C}+\overline{D}) \\
        &=&(A+B+C).(A+\overline{B}+\overline{D}).(\overline{A}+B+D).(\overline{A}+\overline{B}+\overline{C})
    \end{eqnarray*}
    Which in minterms would be, using (2):
    \begin{eqnarray*}
        f_2(A,B,C,D) &= &(\overline{A}\overline{B}C\overline{D})+(\overline{A}\overline{B}CD)+(\overline{A}B\overline{C}\overline{D})+(\overline{A}BC\overline{D})+\\
        &&(A\overline{B}\overline{C}D)+(A\overline{B}CD)+(AB\overline{C}\overline{D})+(AB\overline{C}D)\\
        &=&(A\overline{B}D)+(\overline{A}B\overline{D})+(\overline{A}\overline{B}C)+(AB\overline{C})
    \end{eqnarray*}

    \subsection{\color{purple}Simplify: Karnaugh Map}

    Karnaugh map is a easier way to simplify logic experesion when the functions are too complex or too large to handle, cause Karnaugh map gives a more representative view for a faster analisis for it to simplify. 

    If the simplification is done with minterms, the groups should be of $1$, adding each group in case there is more than 1, and in each group the independent variables would be multiplied.
    \begin{center}
        \begin{Karnaugh}
            %cada 4 es una fila, la col 3 es la 4ta columna y 3fila es la 4 fila
            \contingut{
            0,1,0,1,
            0,0,1,1,
            1,1,0,0,
            1,0,1,0} 
            \implicant{6}{14}{red}
            \implicant{3}{7}{green}
            \implicant{12}{8}{orange}
            \implicantdaltbaix[3pt]{1}{9}{blue}
            %\implicantcantons[2pt]{orange}
            %\implicantcostats{4}{14}{green}
        \end{Karnaugh}
    \end{center}
    
    Now grouping the colour groups we get that the function in minterms would be:
    \begin{eqnarray*}
        f_2(A,B,C,D)&=&(A\overline{B}D)~(Red)\\
        &&+(\overline{A}B\overline{D})~(Blue)\\
        &&+(\overline{A}\overline{B}C)~(Orange)\\
        &&+(AB\overline{C})~(Green)
    \end{eqnarray*}
 
    The same method could be done with maxterms; grouping $O$, multipling groups in case there is more than 1, and in each group the independent variables would be added.
    
    \subsection{\color{purple}Logic Circuit: AND, OR and NOT}

    Using the logic gates AND, OR and NOT the simplify version of the function could be represented in the figure below:

    \begin{figure}[htb!]
        \centering
        \includegraphics[scale=0.45]{normallogic.png}
        \caption{Logic circuit using AND, OR and NOT gates}
        \label{fig:normllogic}
    \end{figure}
    
    \pagebreak

    \subsection{\color{purple}Logic Circuit: NAND}

    All the gates could be equivalent to a combination of NAND or NOR gates. Therefore, the simplify function can be drawn as the next figure:

    \begin{figure}[h!]
        \centering
        \includegraphics[scale=0.45]{nandlogic.png}
        \caption{Logic circuit using NAND gates}
        \label{fig:nandlogic}
    \end{figure}

%////////////////////////////////////////////////////////////////////// EXCERCISE 3 /////////////////////////////////////////////////////////////////////////////
\section{\color{olive}Excercise 3: Four-Entry Encoder and Demux using Verilog}
Implement the following modules in Verilog:
\begin {itemize}
\item 4 inputs \textsc{encoder}
\item 4 outputs \textsc{demux}
\end {itemize}

\subsection{\color{purple}4-Input ENCODER}
\subsubsection{\color{orange}Description}
An encoder is an Application-Specific Integrated Circuit (ASIC) that converts information. In this case it receives a signal from  a 4-bit input and returns the position of the Most Significant Bit that is currently on.
\subsubsection{\color{orange}Code Implementation}
The Code Implementation of both the Module and its testbench can be found in their respective directories.
\subsubsection{\color{orange}Module Tests}
Results of the Testbench:
\begin{center}
\begin{tabular}{|c|c|c|}
\hline
Input&Output&Value\\
\hline
0001&00&0\\
0010&01&1\\
0100&10&2\\
1000&11&3\\
0011&01&1\\
0101&10&2\\
1001&11&3\\
0110&10&2\\
1010&11&3\\
1100&11&3\\
\hline
\end{tabular}
\\\vspace{12pt}
Table 1.1.3 \textsc{encoder} Testbench Results
\end{center}

\subsubsection{\color{orange}Conclusions}
The module is working as expected, where it is taking only the Most Significant Bit as the value to be encoded.

\subsection{\color{purple}4-output DEMUX}

\subsubsection{\color{orange}Description}
A DEMUX is an ASIC which receives an input signal and a selector signal. The selector signal determines through which output port the input signal is sent.

\subsubsection{\color{orange}Code Implementation}
The Code implementation for the \textsc{demux} can be found in its  corresponding folder.

\subsubsection{\color{orange}Module Tests}

\begin{center}
\begin{tabular}{|c|c|c|c|c|c|}
\hline
Input&Selector&Out\_0&Out\_1&Out\_2&Out\_3\\
\hline
1&0&1&0&0&0\\
1&1&0&1&0&0\\
1&2&0&0&1&0\\
1&3&0&0&0&1\\
\hline
\end{tabular}
\\\vspace{12pt}
Table 1.2.3 \textsc{demux} Testbench Results
\end{center}

\subsubsection{\color{orange}Conclusions}
The module is works as expected.

\end{document}


%Información usada:

% beginners guide to latex:
% http://www.docs.is.ed.ac.uk/skills/documents/3722/3722-2014.pdf

% fixed point mathematics:
%http://fileadmin.cs.lth.se/cs/Education/EDA075/notes/mgh_appA_fixed.pdf