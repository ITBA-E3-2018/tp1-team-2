%% LyX 2.2.3 created this file.  For more info, see http://www.lyx.org/.
%% Do not edit unless you really know what you are doing.
\documentclass[english]{article}
\usepackage[T1]{fontenc}
\usepackage[latin9]{inputenc}
\usepackage{geometry}
\geometry{verbose,tmargin=3cm,bmargin=3cm,lmargin=3cm,rmargin=3cm,headheight=3cm,headsep=3cm}
\usepackage{float}

\makeatletter

%%%%%%%%%%%%%%%%%%%%%%%%%%%%%% LyX specific LaTeX commands.
%% Because html converters don't know tabularnewline
\providecommand{\tabularnewline}{\\}

%%%%%%%%%%%%%%%%%%%%%%%%%%%%%% User specified LaTeX commands.
\usepackage{babel}






\usepackage{babel}




\usepackage{babel}


\makeatother

\usepackage{babel}
\begin{document}

\section{Exercise 6: ALU Implementation}

For this exercise we were asked to implement a 4 bit Aritmetic Logic
Unit (ALU). The operations we had to develop were SUM, SUBSTRACTION,
AND, OR, NOT, XOR, two's complement and shift left. For this, we decided
to create a module responsible of adding 2 bits, and as output, it
returned 2 bits, one bit for the answer, and another for the carry
bit. By using this module, we decided now, to create a secondary module,
responsible for adding 3 bits. This decision gave us al lot of simplification
in the development of the module SUM for 4 bits. As you can see in
the code ``sum.v'' found in the folder src, we commented the previous
development without the module sum3Bits, and the new development with
the module sum3Bits.

For the substaction, we decided to re-use the module created on exercise
4 of two's complement, and utilizing it correctly with the module
SUM, we had our SUBSTRACTION module. For the operations AND, OR, NOT,
XOR we chosen to use the predefined modules provided by verilog and
utilize them bitwise.

\subsection{Definiciones}

Esta unidad aritmetica l�gica se implemento con 2 acumuladores (que
llamaremos A y B en este orden) de 4 bits cada uno, tres bits de operaciones,
4 bits del acumulador de salida (que llamaremos Acumulador C) y un
bit de carry, ordenados en el orden en que fueron mencionados. Para
seleccionar que operacion se desea hacer, se deben encender los 3
bits de operaciones de la siguiente manera:

\begin{table}[H]
\begin{centering}
\begin{tabular}{|c|c|c|c|}
\hline 
 & Bit 0  & Bit 1  & Bit 2\tabularnewline
\hline 
\hline 
AND  & 0  & 0  & 0\tabularnewline
\hline 
NOT  & 0  & 0  & 1\tabularnewline
\hline 
OR  & 0  & 1  & 0\tabularnewline
\hline 
XOR  & 0  & 1  & 1\tabularnewline
\hline 
SHIFT LEFT  & 1  & 0  & 0\tabularnewline
\hline 
SUM  & 1  & 0  & 1\tabularnewline
\hline 
SUBSTRACTION  & 1  & 1  & 0\tabularnewline
\hline 
TWO'S COMPLEMENT  & 1  & 1  & 1\tabularnewline
\hline 
\end{tabular}
\par\end{centering}
\caption{Representacion de operaciones en los bits.}
\end{table}
\begin{itemize}
\item AND: Realiza una operacion AND bit a bit entre acumuladores A y B
y la devuelve en el acumulador C, a su vez, el carry se devuelve en
cero. 
\item NOT: Realiza una operacion logica NOT bit a bit del acumulador A y
devuelve la respuesta en el acumulador C, a su vez, el carry queda
en cero. 
\item OR: Realiza una operacion OR bit a bit entre acumuladores A y B y
la devuelve en el acumulador C, a su vez, el carry se devuelve en
cero. 
\item XOR: Realiza una operacion XOR bit a bit entre acumuladores A y B
y la devuelve en el acumulador C, a su vez, el carry se devuelve en
cero. 
\item SHIFT LEFT: Se encarga de mover cada bit del acumulador A un espacio
a la derecha e inserta un cero al bit menos significativo, la respuesta
la devuelve en el acumulador C y el carry valdra 0 o 1 dependiendo
del bit mas significativo de A 
\item SUM: Se encarga de hacer una suma numerica de los valores decimales
de los acumuladores A y B y devuelve el resultado (en codigo binario)
en el acumulador C. Dependiendo si es representable el resultado de
la suma en un nibble se encendera o no el bit Carry. 
\item SUBSTRACTION: Se encarga de hacer una resta numerica de los valores
decimales de los acumuladores A y B y devuelve el resultado (en codigo
binario) en el acumulador C. El carry se devuelve en 1. 
\item TWO'S COMPLEMENT: Se encarga de hacer el complemento a 2 del acumulador
A y devuelve el resultado en el acumulador C. El carry se devuelve
en cero. 
\end{itemize}

\end{document}
