%% LyX 2.2.3 created this file.  For more info, see http://www.lyx.org/.
%% Do not edit unless you really know what you are doing.
\documentclass[english]{article}
\usepackage[T1]{fontenc}
\usepackage[latin9]{inputenc}
\usepackage{geometry}
\geometry{verbose,tmargin=3cm,bmargin=3cm,lmargin=3cm,rmargin=3cm,headheight=3cm,headsep=3cm}

\makeatletter
%%%%%%%%%%%%%%%%%%%%%%%%%%%%%% User specified LaTeX commands.
\usepackage{babel}


\makeatother

\usepackage{babel}
\begin{document}

\section{Ejercicio 6: Implementaci�n de una ALU}
\noindent \begin{flushleft}
Para este ejercicio se pidi� implementar una ALU de 4 bits con las
operaciones SUMA, RESTA, AND, OR, NOT, XOR, Complemento a dos y Shift
Left. Para esto se decidi� crear un modulo encargado de sumar 2 bits,
este a su vez devuelve un bit de respuesta y un bit de carry. Utilizando
ese modulo, se procedi� a hacer un nuevo modulo que haga una suma
de 3 bits. Esta decisi�n simplific� mucho el desarollo del modulo
de suma de la ALU de 4 bits, ya que como se puede ver en el c�digo
``sum.v'' encontrado en la carpeta src, se puede ver comentado dentro
del c�digo como fue nuestro desarrollo sin la funci�n sum3Bits y como
quedo finalmente el c�digo con la implementacion de sum3Bits. 
\par\end{flushleft}

\noindent \begin{flushleft}
Para el caso de la resta se decidi� utilizar el modulo hecho en el
ejercicio 4 del complemento a 2 y as�, se utiliz� en conjunto ese
modulo con el modulo suma. 
\par\end{flushleft}

\noindent \begin{flushleft}
En el caso de las operaciones AND, OR, NOT y XOR, se opto por utilizar
las funciones predefinidas por verilog y reutilizarlas bit a bit. 
\par\end{flushleft}

\subsection{Definiciones}

Para la suma, se definio un bit extra llamado carry que justamente
se enciende en caso de que la suma se pase del rango m�ximo representable
por cuatro bits. Por otro lado, en el caso del Shift left, se definio
que se ingresa un 0 al bit menos significativo del nibble en cuesti�n
y el bit m�s significativo previo al shift, se devuelve como carry.
\end{document}
