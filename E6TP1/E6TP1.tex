

\section{\color{olive}Excercise 6: ALU Implementation}

For this exercise we were asked to implement a 4 bit Aritmetic Logic
Unit (ALU). The operations we had to develop were SUM, SUBSTRACTION,
AND, OR, NOT, XOR, two's complement and shift left. In order to do this, we decided
to create a module responsible of adding 2 bits, and as output, it
returned 2 bits, one bit for the answer, and another for the carry
bit. By using this module, we then decided to create a secondary module,
responsible for adding 3 bits. This decision gave us al lot of simplification
in the development of the module SUM for 4 bits. As you can see in
the code "sum.v'' found in the folder src, we commented the previous
development without the module sum3Bits, and the new development with
the module sum3Bits.

For the substaction, we decided to re-use the module created on exercise
4 of two's complement, and utilizing it correctly with the module
SUM, we had our SUBSTRACTION module. For the operations AND, OR, NOT,
XOR we chosen to use the predefined modules provided by verilog and
utilize them bitwise.

\subsection{\color{purple}Definitions}

In this Aritmetic Logic Unit, we implemented with two accumulators
(that we will call A and B), each one of four bits, three operational
bits, four bits for the output accumulator (that we will call accumulator
C) and one carry bit, ordered in the way they were mentioned.

To select the operation you want to make, you should turn the three
operational bits in the following way:

\begin{table}[h!]
\begin{centering}
\begin{tabular}{|c|c|c|c|}
\hline 
 & Bit 0  & Bit 1  & Bit 2\tabularnewline
\hline 
\hline 
AND  & 0  & 0  & 0\tabularnewline
\hline 
NOT  & 0  & 0  & 1\tabularnewline
\hline 
OR  & 0  & 1  & 0\tabularnewline
\hline 
XOR  & 0  & 1  & 1\tabularnewline
\hline 
SHIFT LEFT  & 1  & 0  & 0\tabularnewline
\hline 
SUM  & 1  & 0  & 1\tabularnewline
\hline 
SUBSTRACTION  & 1  & 1  & 0\tabularnewline
\hline 
TWO'S COMPLEMENT  & 1  & 1  & 1\tabularnewline
\hline 
\end{tabular}
\par\end{centering}
\caption{Representation of operational bits}
\end{table}
\begin{itemize}
\item AND: Performs an AND operation bitwise between acummulators A and
B and returns it on accumulator C, meanwhile, the carry bit is left
to zero.
\item NOT: Performs a logic NOT operation bitwise between acummulators A
and B, and returns the answer in the accumulator C. The carry bit
stays as zero
\item OR: Performs a logic OR operation bitwise between accumulators A and
B, and returns the answer in the accumulator C. The carry bit stays
as zero.
\item XOR: Performs a logic XOR operation bitwise between accumulators A
and B, and returns the answer in the accumulator C. The carry bit
stays as zero.
\item SHIFT LEFT: It manages to move every bit in accumulator A, one space
left, and insterts a logic zero to the less significant bit. The answer
is given in the C acccumulator and the carry bit will become 0 or
1 dependig on the most significant bit of A.
\item SUM: Performs a numeric sum of the binary values of accumulator A
and B and the answer is given in accumulator C. Depending on the overflow,
the carry bit will become 1 or 0.
\item SUBSTRACTION: Performs a numeric substraction of the binary values
of accumulator A and B and the answer is given in accumulator C. The
carry bit will become 1.
\item TWO'S COMPLEMENT: Performs a two's complement of the binary value
of accumulator A and the answer is given in accumulator C. The carry
bit will become 0.
\end{itemize}

