\section {\color{olive}Excercise 5: BCD Format Adder}
Implement a module that receives as inputs two one-digit numbers in Binary-Coded-Decimals (BCD) format and outputs a two-digit number in BCD format.

\subsection{\color{purple}Design Considerations}
\begin{itemize}
\item BCD digits are comprised of 4 bits with a range of integer values between 0 and 9. Any value outside that range should be considered an error.
\item It needs to make a simple addition. Given that the maximum value of the sum is 18, the result will be a 5-bit integer.
\item The value of the sum must be returned in BCD format, so the 5-bit integer needs to be split back into 2 BCD digits.
\end{itemize}

Given these conditions, the module will need:
\begin{itemize}
\item 2 4-bit input ports
\item 2 4-bit output ports
\item 1 \textsc{error} register
\end{itemize}

\subsection{\color{purple}Code Implementation}
The BCD Adder was composed of the following modules:
\begin{itemize}
\item 2 BCD format "filters"
\item 1 4-bit numbers adder
\item 1 BCD format "decoder"
\end{itemize}

The code implementation for each of the modules can be found in their respective folders.

\subsection{\color{purple}Module Testbench}
Testbench results for each submodule can be found in the the directory /E5TP1/tests.txt.\\
Testbench results for the main module can be found in the directory /E5TP1/final.txt.

\subsection{\color{purple}Conclusions}
Each sub-module is working as intended.
